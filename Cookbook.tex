%Author: Christopher Sara
% 
%
%
%Copyright 2014 



\documentclass[12pt]{article}
%a4paper,
%twoside,


% encoding, font, language
\usepackage[T1]{fontenc}
\usepackage[latin1]{inputenc}
\usepackage{lmodern}
\usepackage[ngerman, english]{babel}
\usepackage{gensymb}
\usepackage{nicefrac}
\usepackage{titlesec}

\usepackage[
    handwritten,
    nowarnings,
    %myconfig
]{xcookybooky}

\usepackage{blindtext}    % only needed for generating test text

\definecolor{mygreen}{rgb}{0,.5,0}
\DeclareRobustCommand{\textdegree}{\ensuremath{^{\circ}\mathrm{F}}}

\renewcommand{\familydefault}{\sfdefault}


%----------------------------------------------------------------------------------------
%	SET RECIPE COLORS
%----------------------------------------------------------------------------------------
\setRecipeColors
{%
    recipename = mygreen,
    ing = blue,
    inghead = blue,
    prep,
    prephead,
    hint,
    hinthead,
}


%----------------------------------------------------------------------------------------
%	SET RECIPE HEADLINE NAMES
%----------------------------------------------------------------------------------------


\setHeadlines{
inghead = Ingredients,
prephead = Directions,
preparationtime = Preparation,
bakingtime = Cook,
hinthead = Hint,
calory = Calories, % deal w/ it the dude that wrote this is german
portion = Servings,
portionvalue =,
continuationhead = Continuation,
continuationfoot = Continuation on next page,
source = By
}

%----------------------------------------------------------------------------------------
%	SECTION OPTIONS
%----------------------------------------------------------------------------------------

\setcounter{secnumdepth}{1}

\renewcommand*{\recipesection}[2][]
{%
    \subsubsection[#1]{#2}
}
\renewcommand{\subsectionmark}[1]
{% no implementation to display the section name instead
}


%%%%%%%%%%
% hyperref
\usepackage{hyperref}    % must be the last package
\hypersetup{%
    pdfauthor            = {Christopher Sara},
    pdftitle             = {Cookbook!},
    pdfsubject           = {Recipes},
    pdfkeywords          = {recipes},
    pdfstartview         = {FitV},             
    pdfview              = {FitH},
    pdfpagemode          = {UseNone}, % Options; UseNone, UseOutlines
    bookmarksopen        = {true},
    pdfpagetransition    = {Glitter},
    colorlinks           = {true},
    linkcolor            = {black}, 
    urlcolor             = {black}
    citecolor            = {black}, 
    filecolor            = {black},
}
% hyperref
%%%%%%%%%%



\begin{document}

\title{Cookbook v0.1}
\author{Christopher Sara}
\maketitle

\newpage

\tableofcontents

%\vspace{9em}

\newpage

\section{Breakfast}
\newpage

\begin{recipe}
[ % 
    preparationtime = {\unit[30]{min}},
    bakingtime={\unit[45]{min}},
    bakingtemperature={\protect\bakingtemperature{fanoven=\unit[350]{\textdegree}F}},
    %bakingtemperature={\protect\bakingtemperature{fanoven=\unit[230]{\textdegree}F, topbottomheat=\unit[195]{\textdegree}F, topheat=\unit[195]{\textdegree}F, gasstove=Stove 2}},
    portion = {\portion{4--5}},
    source = Jennifer Sara
]
{Egg Casserole}
    
    %\graph
    %{% pictures
       % small=pic/glass,     % small picture
        %big=pic/ingredients  % big picture
    %}
    
    \ingredients
    {%
    	1 & lb.  & breakfast sausage \\
    	6--8 &   & eggs \\
    	2 & c  & milk \\
        1 & tsp & dry mustard\\
        4--5 & slices & cubed white bread \\
        1 & c & cheese\\
        1 & tsp & salt\\               
    }
    
    \preparation
    {%
    	\step Brown sausage in a pan and drain fat.
        \step Beat eggs in a large bowl. 
        \step Add milk, mustard, and bread. Stir.
        \step Add the cheese and sausage.
        \step Place mixture into a greased 13x9 pan.
        \step Preheat oven to \unit[350]{\textdegree}F. Bake for 45 minutes. 
    }
    
    \hint{Substitute the sausage for your preferred vegetables or use both.}
\end{recipe}
\newpage

\section{Dinner}
\begin{recipe}
[ % 
    preparationtime = {\unit[10]{min}},
    bakingtime={\unit[20-25]{min}},
   % bakingtemperature={\protect\bakingtemperature{fanoven=\unit[300]{\textdegree}F}},
    %bakingtemperature={\protect\bakingtemperature{fanoven=\unit[230]{\textdegree}F, topbottomheat=\unit[195]{\textdegree}F, topheat=\unit[195]{\textdegree}F, gasstove=Stove 2}},
    portion = {\portion{5-6 Servings}},
    source = Jennifer Sara
]
{Chicken Wings \& Rice}
    
    %\graph
    %{% pictures
       % small=pic/glass,     % small picture
        %big=pic/ingredients  % big picture
    %}
    
    \ingredients
    {%
    	\
        2 & pckgs & wings\\
	$3 \nicefrac{1}{2}$ & c  & chicken broth \\
	$1 \nicefrac{1}{2}$ & c  & white rice, uncooked \\
	$\nicefrac{3}{4}$ & c  & flour \\
	$\nicefrac{1}{2}$ & c  & white rice, uncooked \\
	$\nicefrac{1}{2}$ & tsp.  & salt \\
	$\nicefrac{1}{2}$ & tsp. & pepper \\
	$\nicefrac{1}{2}$ & tsp. & thyme\\
	$\nicefrac{1}{2}$ & cbe  & butter \\
        2 & stalks & celery\\
        2 & & green onions, sliced\\              
    }
    
    \preparation
    {%
    	\step Wash chicken and dry with paper towel.
        \step Combine flour, salt, pepper, and thyme in a paper or Ziplc bag.
        \step Add chicken and coat with mixture.
        \step Melt butter in skillet over medium-high heat.
        \step Shake off excess coating from chicken and brown with butter until browned and cooked.
        \step Remove chicken to plate and keep warm.
        \step Add celery, onion, and rice. Stir until brown.
        \step Add broth and return chicken and accumulated juiced to skillet.
        \step Cover tightly and simmer with lid on for 20-25 minutes until rice is cooked. Stir occasionally.
    }
    
    \hint{Can be served with toasted almond on top and/or cranberry sauce}

\end{recipe}
\newpage
\begin{recipe}
[ % 
    preparationtime = {\unit[15]{min}},
    bakingtime={\unit[20]{min}},
   bakingtemperature={\protect\bakingtemperature{fanoven=\unit[375]{\textdegree}F}},
    %bakingtemperature={\protect\bakingtemperature{fanoven=\unit[230]{\textdegree}F, topbottomheat=\unit[195]{\textdegree}F, topheat=\unit[195]{\textdegree}F, gasstove=Stove 2}},
    portion = {\portion{5-6 Servings}},
    source = Christopher Sara
]
{Garlic Shrimp Rigatoni}
    
    %\graph
    %{% pictures
       % small=pic/glass,     % small picture
        %big=pic/ingredients  % big picture
    %}
    
    \ingredients
    {%    	
        1  & lb.  & raw shrimp (tail-off) \\
    	4 & cloves  & garlic \\
    	1 & tbls  & butter \\
        6 & oz & Rigatoni\\ 
        1 & can & fire-roasted tomatoes\\              
        16 & oz & alfredo\\
        $\nicefrac{1}{4}$ & cup  & Parmesan cheese \\         
    }
    
    \preparation
    {%
    	\step Preheat oven to 375 {\textdegree}F.  
    	\step Cook pasta in a pot.      
    	\step In a skillet, melt 1 tablespoon of butter and sauté garlic.  Add shrimp and cook. Set aside.
    	\step Drain pasta and add to baking pan. 
    	\step Add tomatoes, alfredo, and cooked shrimp to the pan. Top with Parmesan if desired. 
    	\step Put in oven for about 10-15 minutes or until cheese is golden brown on top.
    }
    
    \hint{A 13x9 fits this dish perfectly.}

\end{recipe}
\newpage

\section{Salads}
\newpage

\section{Desserts}
\begin{recipe}
[ % 
    preparationtime = {\unit[1]{h}},
    bakingtime={\unit[1]{h}},
    bakingtemperature={\protect\bakingtemperature{fanoven=\unit[300]{\textdegree}F}},
    %bakingtemperature={\protect\bakingtemperature{fanoven=\unit[230]{\textdegree}F, topbottomheat=\unit[195]{\textdegree}F, topheat=\unit[195]{\textdegree}F, gasstove=Stove 2}},
    portion = {\portion{10-12 Slices}},
    source = Jennifer Sara
]
{Banana Bread}
    
    %\graph
    %{% pictures
       % small=pic/glass,     % small picture
        %big=pic/ingredients  % big picture
    %}
    
    \ingredients
    {%
    	$\nicefrac{1}{2}$ & c  & butter (room temp.) \\
    	$\nicefrac{1}{2}$ & c  & sugar \\
    	$\nicefrac{1}{2}$ & c  & brown sugar \\
        1 & tsp & vanilla\\
        2 & & eggs \\
        2 & c & flour\\
        1 & tsp & baking soda\\ 
        2 & overripe & bananas \\
	$\nicefrac{1}{2}$ & c & walnuts and/or \\ & chocolate chips \\               
    }
    
    \preparation
    {%
    	\step Preheat oven to 300 {\textdegree}F.
        \step Mix butter, sugar, and vanilla into a medium sized bowl. 
        \step While stirring, add in one egg at a time.
        \step Add baking soda.
        \step Stir in flour $\nicefrac{1}{2}$ c at a time until completely mixed in.
        \step You may add in $\nicefrac{1}{2}$ c and/or $\nicefrac{1}{2}$ c chocolate chips if you like.
        \step Place dough in well-greased or non-stick bread pan. Smooth the top.
        \step Put in oven for 65--70 minutes.  Test near the end to make sure dough has fully baked.
        \step Let bread cool in the pan for 10 minutes then put the bread on a rack to cool.        
    }
    
    \hint{Be sure to allow bread cool well before slicing.}

\end{recipe}

\begin{recipe}
[ % 
    preparationtime = {\unit[10]{min}},
    bakingtime={\unit[30]{min}},
    bakingtemperature={\protect\bakingtemperature{fanoven=\unit[325]{\textdegree}F}},
    %bakingtemperature={\protect\bakingtemperature{fanoven=\unit[230]{\textdegree}F, topbottomheat=\unit[195]{\textdegree}F, topheat=\unit[195]{\textdegree}F, gasstove=Stove 2}},
    portion = {\portion{12 bars}},
    source = Jennifer Sara
]
{Cookie Dough Cheesecake Bars}
    
    %\graph
    %{% pictures
       % small=pic/glass,     % small picture
        %big=pic/ingredients  % big picture
    %}
    
    \ingredients
    {%
    	%$\nicefrac{1}{2}$ & c  & butter (room temp.) \\
    	%$\nicefrac{1}{2}$ & c  & sugar \\
    	%$\nicefrac{1}{2}$ & c  & brown sugar \\
        % 1 & tsp & vanilla\\             
    }
    
    \preparation
    {%
    	\step Preheat oven to 325 {\textdegree}F.       
    }
    
    \hint{Be sure to allow bread cool well before slicing.}

\end{recipe}
\newpage

\begin{recipe}
[ % 
    preparationtime = {\unit[10]{min}},
    bakingtime={\unit[30]{min}},
    bakingtemperature={\protect\bakingtemperature{fanoven=\unit[300]{\textdegree}F}},
    %bakingtemperature={\protect\bakingtemperature{fanoven=\unit[230]{\textdegree}F, topbottomheat=\unit[195]{\textdegree}F, topheat=\unit[195]{\textdegree}F, gasstove=Stove 2}},
    portion = {\portion{One 13x9}},
]
{Snickerdoodle Bars}
    
    %\graph
    %{% pictures
       % small=pic/glass,     % small picture
        %big=pic/ingredients  % big picture
    %}
    
    \ingredients
    {%
    	$2 \nicefrac{2}{3}$ & c  & all-purpose flour \\
        2 & c & dark brown sugar\\
        1 & c & \textbf{unsalted} butter, room temp. \\
        2 &  & eggs\\
        3 & tbls & white sugar\\ 
        1 & tbls & Vanilla \\
	3 & tsps & groud cinnamon \\
	2 & tsps & baking powder \\
	1 & tsps & salt \\                
    }
    
    \preparation
    {%
    	\step Preheat oven to 350 {\textdegree}F. Lightly grease a 13x9 baking pan.
        \step In a medium bowl, whisk together the flour, baking powder, and salt.  
        \step In a large mixing bowl, beat together the butter, brown sugar, eggs, and vanilla until smooth. 
        \step Add the flour mixture to the egg mixture and beat until well blended.
        \step Spread the batter evenly in the prepared pan with a spoon or rubber scraper.
        \step In a small bowl, combine the white sugar and cinnamon. 
        \step Sprinkle the mixture evenly over the batter in the baking pan. Spread around top of batter.
        \step Bake for 25-30 minutes or until the surface springs back when gently pressed. 
        \step Remove from oven and let cool it slightly on a wire rack.   
    }
    
    \hint{If you use salted butter, you're gonna have a bad time.}

\end{recipe}

\end{document}