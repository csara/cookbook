\documentclass[12pt]{article}
%a4paper,
%twoside,


% encoding, font, language
\usepackage[T1]{fontenc}
\usepackage[latin1]{inputenc}
\usepackage{lmodern}
\usepackage[ngerman, english]{babel}
\usepackage{gensymb}
\usepackage{nicefrac}
\usepackage{titlesec}

\usepackage[
    handwritten,
    nowarnings,
    %myconfig
]
{xcookybooky}

\usepackage{blindtext}    % only needed for generating test text

\definecolor{mygreen}{rgb}{0,.5,0}
\DeclareRobustCommand{\textdegree}{\ensuremath{^{\circ}\mathrm{F}}}

\renewcommand{\familydefault}{\sfdefault}


%----------------------------------------------------------------------------------------
%	SET RECIPE COLORS
%----------------------------------------------------------------------------------------
\setRecipeColors
{%
    recipename = mygreen,
    ing = blue,
    inghead = blue,
    prep,
    prephead,
    hint,
    hinthead,
}


%----------------------------------------------------------------------------------------
%	SET RECIPE HEADLINE NAMES
%----------------------------------------------------------------------------------------


\setHeadlines{
inghead = Ingredients,
prephead = Directions,
preparationtime = Preparation,
bakingtime = Cook,
hinthead = Hint,
calory = Calories, % deal w/ it the dude that wrote this is german
portion = Servings,
portionvalue =,
continuationhead = Continuation,
continuationfoot = Continuation on next page,
source = By
}

%----------------------------------------------------------------------------------------
%	SECTION OPTIONS
%----------------------------------------------------------------------------------------

\setcounter{secnumdepth}{1}

\renewcommand*{\recipesection}[2][]
{%
    \subsubsection[#1]{#2}
}
\renewcommand{\subsectionmark}[1]
{% no implementation to display the section name instead
}


%%%%%%%%%%
% hyperref
\usepackage{hyperref}    % must be the last package
\hypersetup{%
    pdfauthor            = {Christopher Sara},
    pdftitle             = {Cookbook v0.1},
    pdfsubject           = {Recipes},
    pdfkeywords          = {recipes},
    pdfstartview         = {FitV},             
    pdfview              = {FitH},
    pdfpagemode          = {UseNone}, % Options; UseNone, UseOutlines
    bookmarksopen        = {true},
    pdfpagetransition    = {Glitter},
    colorlinks           = {true},
    linkcolor            = {black}, 
    urlcolor             = {black}
    citecolor            = {black}, 
    filecolor            = {black},
}
% hyperref
%%%%%%%%%%



\begin{document}

\title{Cookbook v0.1}
\author{Christopher Sara}
\maketitle

\newpage

\tableofcontents

%\vspace{9em}

\newpage

\section{Breakfast}
\newpage


\subsection{Breads}
\newpage

\begin{recipe}
[ % 
    preparationtime = {\unit[1]{h}},
    bakingtime={\unit[1]{h}},
    bakingtemperature={\protect\bakingtemperature{fanoven=\unit[300]{\textdegree}F}},
    %bakingtemperature={\protect\bakingtemperature{fanoven=\unit[230]{\textdegree}F, topbottomheat=\unit[195]{\textdegree}F, topheat=\unit[195]{\textdegree}F, gasstove=Stove 2}},
    portion = {\portion{10-12 Slices}},
    source = Jennifer Sara
]
{Banana Bread}
    
    %\graph
    %{% pictures
       % small=pic/glass,     % small picture
        %big=pic/ingredients  % big picture
    %}
    
    \ingredients
    {%
    	$\nicefrac{1}{2}$ & c  & butter (room temp.) \\
    	$\nicefrac{1}{2}$ & c  & sugar \\
    	$\nicefrac{1}{2}$ & c  & brown sugar \\
        1 & tsp & vanilla\\
        2 & & eggs \\
        2 & c & flour\\
        1 & tsp & baking soda\\ 
        2 & overripe & bananas \\
	$\nicefrac{1}{2}$ & c & walnuts and/or \\ & & chocolate chips \\               
    }
    
    \preparation
    {%
    	\step Preheat oven to 300 {\textdegree}F.
        \step Mix butter, sugar, and vanilla into a medium sized bowl. 
        \step While stirring, add in one egg at a time.
        \step Add baking soda.
        \step Stir in flour $\nicefrac{1}{2}$ c at a time until completely mixed in.
        \step You may add in $\nicefrac{1}{2}$ c and/or $\nicefrac{1}{2}$ c chocolate chips if you like.
        \step Place dough in well-greased or non-stick bread pan. Smooth the top.
        \step Put in oven for 65--70 minutes.  Test near the end to make sure dough has fully baked.
        \step Let bread cool in the pan for 10 minutes then put the bread on a rack to cool.        
    }
    
    \hint{Be sure to allow bread cool well before slicing.}

\end{recipe}
\newpage

\subsection{Egg Plates}
\newpage

\begin{recipe}
[ % 
    preparationtime = {\unit[30]{min}},
    bakingtime={\unit[45]{min}},
    bakingtemperature={\protect\bakingtemperature{fanoven=\unit[350]{\textdegree}F}},
    %bakingtemperature={\protect\bakingtemperature{fanoven=\unit[230]{\textdegree}F, topbottomheat=\unit[195]{\textdegree}F, topheat=\unit[195]{\textdegree}F, gasstove=Stove 2}},
    portion = {\portion{4--5}},
    source = Jennifer Sara
]
{Egg Casserole}
    
    %\graph
    %{% pictures
       % small=pic/glass,     % small picture
        %big=pic/ingredients  % big picture
    %}
    
    \ingredients
    {%
    	1 & lb.  & breakfast sausage \\
    	6--8 &   & eggs \\
    	2 & c  & milk \\
        1 & tsp & dry mustard\\
        4--5 & slices & cubed white bread \\
        1 & c & cheese\\
        1 & tsp & salt\\               
    }
    
    \preparation
    {%
    	\step Brown sausage in a pan and drain fat.
        \step Beat eggs in a large bowl. 
        \step Add milk, mustard, and bread. Stir.
        \step Add the cheese and sausage.
        \step Place mixture into a greased 13x9 pan.
        \step Preheat oven to \unit[350]{\textdegree}F. Bake for 45 minutes. 
    }
    
    \hint{Substitute the sausage for your preferred vegetables or use both.}
\end{recipe}
\newpage

\section{Slow Cooked}
\newpage

\section{Salads}
\newpage

\section{Desserts}
\newpage

\end{document}